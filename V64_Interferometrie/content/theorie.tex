\section{Theorie}
\label{sec:Theorie}

%Kohärenz
\subsection{Interferenz und Kohärenz}
\textbf{Interferenz} beschreibt den Effekt der bei der Überlagerung zweier oder mehrerer Wellen auftritt.
Die resultierende Welle entspricht nicht der Summe der Intensitäten der Einzelwellen, sondern der vektoriellen Summe.
Dabei wird die Auslenkungsrichtung der Welle berücksichtigt.
Destruktive Interferenz tritt auf, wenn die Auslenkung am betrachteten Orts- und Zeitpunkt entgegengesetzt gleich groß sind.
Dabei löschen sich beide Wellen aus.
\\
Damit zwei elektromagnetische Wellen miteinander interferieren müssen sie kohärent sein.
\textbf{Kohärenz} heißt, dass eine feste Phasen- und Amplitudenbeziehung zwischen beider Wellen besteht.
Die Kohärenz nimmt ab, wenn die Phase oder Amplitude der Wellen fluktuieren.
Das Interferenzmuster wird zerstört.
\\
Die Phasenbeziehung kann zum einen zeitlich konstant sein, hier spricht man von zeitlicher Kohärenz.
Zum anderen ist eine zeitlich verändernde Phase möglich, die an jedem Raumpunkt konstant ist.
Dies wird als räumliche Kohärenz bezeichnet.

%Polarisation
\subsection{Polarisation}
Grundsätzlich beschreibt die Polarisation optischer Wellen die Auslenkungsrichtung des elekrischen bzw. magnetischen Feldes.
Ist die Auslenkungsrichtung einer Welle konstant, so handelt es sich um \textbf{linear polarisiertes} Licht.
Die Auslenkung liegt in einer Ebene, der sog. Polarisationsebene.
\\
Ändert sich die Auslenkungsrichtung um einen konstanten Wert, so rotiert die Polarisationsebene mit einer konstanten Winkelgeschwindigkeit um den Ausbreitungsvektor ($\vec{k}$).
Licht mit dieser Eigenschaft wird als \textbf{zirkular polarisiertes} Licht bezeichntet.
\\
Das Fresnel-Arago-Gesetz\cite{hecht2018optik} besteht aus vier Aussagen zur Interferenz polarisierten Lichts.
Die für diesen Versuch relevanten Aussagen sind zum einen, dass zwei linear polarisierte Lichtstrahlen nur bei einer parallelen Ausrichtung der Polarisationsebenen interferieren.
Zum anderen interferieren zwei senkrecht zueinander linear polarsierte Lichtstrahlen, wenn sie Teilstrahlen eines Lichtstrahls mit gleicher Polarisation sind und anschließend wieder in dieselbe Polarisationsebene gebracht werden.

%Kontrast
\subsection{Kontrast eines Interferometers}

%n Glas
\subsection{Brechungsindex von Glas}


%n Luft
\subsection{Brechungsindex von Luft}