\section{Theorie}
\label{sec:Theorie}

%Kohärenz
\subsection{Interferenz und Kohärenz}
\textbf{Interferenz} beschreibt den Effekt der bei der Überlagerung zweier oder mehrerer Wellen auftritt.
Die resultierende Welle entspricht nicht der Summe der Intensitäten der Einzelwellen, sondern der vektoriellen Summe.
Dabei wird die Auslenkungsrichtung der Welle berücksichtigt.
Destruktive Interferenz tritt auf, wenn die Auslenkung am betrachteten Orts- und Zeitpunkt entgegengesetzt gleich groß sind.
Dabei löschen sich beide Wellen aus.
\\
Damit zwei elektromagnetische Wellen miteinander interferieren, müssen sie kohärent sein.
\textbf{Kohärenz} heißt, dass eine feste Phasen- und Amplitudenbeziehung zwischen beiden Wellen besteht.
Die Kohärenz nimmt ab, wenn die Phase oder Amplitude der Wellen fluktuieren.
Das Interferenzmuster wird zerstört.
\\
Die Phasenbeziehung kann zum einen zeitlich konstant sein, hier spricht man von zeitlicher Kohärenz.
Zum anderen ist eine zeitlich verändernde Phase möglich, die an jedem Raumpunkt konstant ist.
Dies wird als räumliche Kohärenz bezeichnet.

%Polarisation
\subsection{Polarisation}
\label{sec:polarisation}

Grundsätzlich beschreibt die Polarisation optischer Wellen die Auslenkungsrichtung des elekrischen bzw. magnetischen Feldes der Welle.
Ist die Auslenkungsrichtung einer Welle konstant, so handelt es sich um \textbf{linear polarisiertes} Licht.
Die Auslenkung liegt in einer Ebene, der sog. Polarisationsebene.
\\
Ändert sich die Auslenkungsrichtung um einen konstanten Wert, so rotiert die Polarisationsebene mit einer konstanten Winkelgeschwindigkeit um den Ausbreitungsvektor (Wellenzahlvektor $\vec{k}$).
Licht mit dieser Eigenschaft wird als \textbf{zirkular polarisiertes} Licht bezeichntet.
\\
Das Fresnel-Arago-Gesetz\cite{hecht2018optik} beinhaltet vier Aussagen zur Interferenz von polarisiertem Licht.
Die für diesen Versuch relevanten Aussagen sind zum einen, dass zwei linear polarisierte Lichtstrahlen nur bei einer parallelen Ausrichtung der Polarisationsebenen interferieren.
Zum anderen interferieren zwei senkrecht zueinander linear polarsierte Lichtstrahlen, wenn sie Teilstrahlen eines Lichtstrahls mit gleicher Polarisation sind und anschließend wieder in dieselbe Polarisationsebene gebracht werden.

%Kontrast
\subsection{Kontrast eines Interferometers}
Der Kontrast oder auch Sichtbarkeit ist ein Maß für den Intensitätsunterschied zwischen zwei abgebildeten Punkten.
Im Bezug auf ein Interferometer ist der Kontrast als
\begin{equation}
    \nu = \frac{I_\text{max}-I_\text{min}}{I_\text{max}+I_\text{min}}
    \label{eq:kontrast1}
\end{equation}
definiert.
Dabei beschreibt $I_\text{max}$ die maximale und $I_\text{min}$ die minimale Intensität eines Interferenzmusters.
Liegt das Intensitätsminimum bei Null, so nimmt der Kontrast den maximal möglichen Wert $\nu=1$ an.
Der kleinst mögliche Wert des Kontrasts $\nu=0$ wird für $I_\text{max}=I_\text{min}$ erreicht.
\\
\\
Die Intensität der resultierenden Welle bei einer Überlagerung zweier Wellen mit Amplitude $E_1$ bzw. $E_2$ ist durch
\begin{equation*}
    I \propto \, <|E_1 \cos(\omega t) + E_2 \cos(\omega t + \delta)|^2>
\end{equation*}
mit der Kreisfrequenz der Welle $\omega$ und Phasenverschiebung $\delta$ gegeben.
Die mittlere Intensität eines Lichtstrahls wird über die zeitliche Mittelung $<f(t)> = \frac{1}{T} \int_{t_0}^{t_0 + T} f(t) \mathrm{d}t$ bestimmt.
Beide interferierende Strahlen sind Teilstrahlen des vom Laser ausgesendeten Strahls mit Amplitude $E_0$.
Abhängig von der Einstellung des Polarisationswinkels $\phi$ wird die Amplitude nach
\begin{equation*}
    E_1 = E_0 \cdot \cos(\phi), \qquad E_2 = E_0 \cdot \sin(\phi)
\end{equation*}
aufgeteilt.
Nach einsetzen in die Itensität
\begin{equation*}
    I \propto \, <|E_0 \cos(\phi) \cos(\omega t) + E_0 \sin(\phi)\cos(\omega t + \delta)|^2>
\end{equation*}
und auflösen des Quadrats kann jeweils das zeitliche Mittel der drei resultierenden Terme bestimmt werden.
Mittels Additionstheorem kann die Intensität auf den Ausdruck
\begin{equation*}
    I \propto \, \frac{1}{2} E_0^2 ( 1 + 2 \sin(\phi) \cos(\phi) \cos(\delta) ) 
\end{equation*}
reduziert werden.
Die maximale bzw. minimale Intensität
\begin{equation*}
    I_\text{max/min} = I_\text{ges} (1 \pm 2 \sin(\phi) \cos(\phi))
\end{equation*}
folgt für die Phasenverschiebung $\delta_\text{max} = 2\pi N$ bzw. $\delta_\text{min} = 2\pi (N+1)$, wobei $N \in \mathbf{N_0}$.
\\
Eingesetzt in \autoref{eq:kontrast1} kann der Kontrast
\begin{equation}
    \nu (\phi) = 2 \sin(\phi) \cos(\phi)
    \label{eq:kontrast2}
\end{equation}
in Abhängigkeit der Einstellung des Polarisationsfilters (Polarisationswinkel $\phi$) ausgedrückt werden.

%n Glas
\subsection{Brechungsindex von Glas}
In diesem Versuch soll mithilfe des Interferometers Brechungsindizes verschiedener Materialien ermittelt werden.
Die Lichtgeschwindigkeit im Medium $c_\text{m}$ ist über den materialabhängigen Brechungsindex $n$ durch $c_\text{m} = \frac{c}{n}$ definiert.
Ein Lichtstrahl der durch das Medium geleitet wird, benötigt für den gleichen Weg den ein Lichtstrahl im Vakuum zurücklegt mehr Zeit ($n>1$).
Zwischen dem Lichtstrahl im Medium und im Vakuum existiert eine Phasenverschiebung $\delta$.
\\
Das in diesem Versuch untersuchte Glas befindet sich in einem Doppelglashalter.
Beide Teilstrahlen verlaufen jeweils durch eine Glasplatte, die in einem Winkel von $\theta_0 = \pm \qty{10}{\degree}$ zueinander verkippt sind.
Abhängig vom Winkel des Doppelglashalters $\theta$ durchlaufen die Strahlen unterschiedliche Streckenlängen im Glas.
Die Phasenverschiebung für das Doppelglas ist für kleine Drehwinkel $\theta$ gegeben durch
\begin{equation*}
    \Delta \delta (\theta) = \frac{2\pi}{\lambda_\text{vac}} T \frac{n-1}{2n} \left \{ (\theta+\theta_0)^2 - (\theta-\theta_0)^2 \right \} \, ,
\end{equation*}
wobei $\lambda_\text{vac}$ die Vakuumwellenlänge des vom Laser ausgesandten Strahls und $T$ die Dicke der Glasplatte angibt.
Die Phasenverschiebung zwischen zwei interferierenden Wellen ist unmittelbar mit der Anzahl der Interferenzmaxima bzw. -minima über $M = \frac{\Delta \delta}{2\pi}$ verknüpft.
Weiter folgt für den Brechungsindex von dem Doppelglas
\begin{equation}
    n = \frac{1}{1- \frac{M \lambda_\text{vac}}{2 T \theta \theta_0}} \, .
    \label{eq:n_glas}
\end{equation}

%n Luft
\subsection{Brechungsindex von Luft}
Ein Lichtstrahl der eine evakuierte Gaszelle mit Länge $L$ durchläuft erfährt eine Phasenverschiebung
\begin{equation*}
    \Delta \delta =  \frac{2 \pi L}{\lambda_\text{vac}} (n-1)
\end{equation*}
gegenüber einem Lichtstrahl im Medium Luft mit Brechungsindex $n$.
Die Phasenverschiebung wird erneut durch die Anzahl der Interferenzmaxima bzw. -minima ausgedrückt und es folgt für den Brechungsindex von Luft
\begin{equation}
    n = \frac{M \lambda_\text{vac}}{L} + 1 \, .
    \label{eq:n_luft}
\end{equation}
Das Lorentz-Lorenz-Gesetz setzt den Brechungsindex mit der Polarisierbarkeit des Mediums in Beziehung.
Für den Brechungsindex von Gas gilt
\begin{equation*}
    \frac{n^2-1}{n^2+1} = \frac{A p}{R T}
\end{equation*}
mit Temperatur $T$ und Druck $p$ des Gases, der allgemeinen Gaskonstante $R$ und Molrefraction $A$.
Näherungsweise für Gase mit $n \approx 1$ (Taylorentwicklung 1. Ordnung um $n=1$) kann der Brechungsindex über
\begin{equation}
    n \approx \frac{3}{2} \frac{A p}{R T} + 1
    \label{eq:n_luft_lorentz}
\end{equation}
bestimmt werden.