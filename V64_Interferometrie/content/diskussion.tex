\section{Diskussion}
\label{sec:Diskussion}

Der berechnete Brechungsindex von Glas beträgt $n_\text{Glas} = \qty{1.549(8)}{}$.
Die genaue Zusammensetzung des verwendeten Glases ist nicht bekannt, allerdings liegt der Brechungsindex von Glas üblicherweise im Bereich $\qty{1.46}{}$ bis $\qty{1.65}{}$. \cite{index}
Somit liegt der bestimmte Wert im zu erwartenden Bereich.
\\
Der Brechungsindex von Luft ist $n_\text{Luft} = \qty{1.000292}{}$. 
Der im Experiment bestimmte Wert bei Normatmosphäre liegt bei $n_\text{Glas} = \qty{1.0002757(29)}{}$ und weicht somit um weniger als $\qty{1}{\percent}$ ab.
\\
Dahingehend erfüllen die Ergbnisse die Erwartungen. 
Auffallend ist, dass alle drei Messreihen zur Berechnung des Brechungsindices von Luft sehr ähnlich sind.
Dadurch ist das gemittelte Ergebnis nur wenig durch Schwankung betroffen.