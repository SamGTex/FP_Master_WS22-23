\section{Diskussion}
\label{sec:Diskussion}

Der berechnete Brechungsindex von Glas beträgt $n_\text{Glas} = \qty{1.549(8)}{}$.
Die genaue Zusammensetzung des verwendeten Glases ist nicht bekannt, allerdings liegt der Brechungsindex von Glas üblicherweise im Bereich $\qty{1.46}{}$ bis $\qty{1.65}{}$.\cite{index}
Somit liegt der bestimmte Wert im zu erwartenden Bereich.
\\
Der Brechungsindex von Luft $n_\text{Luft} = \qty{1.000292}{}$ wird der Literatur entnommen. \cite{index}
Bei Normatmosphäre liegt der experimentell ermittelte Wert bei $n_\text{Glas} = \qty{1.0002764(34)}{}$.
Im Bezug auf die signifikanten Stellen des Brechungsindex beträgt die Abweichung vom Literaturwert $\qty{4.45}{\percent}$.
\\
Dahingehend erfüllen die Ergbnisse die Erwartungen. 
Auffallend ist, dass alle drei Messreihen zur Berechnung des Brechungsindex von Luft ähnliche Werte aufweisen.
Dadurch ist das gemittelte Ergebnis nur wenig von Schwankungen betroffen.
Dies zeigt, dass das Interferometer stabil gegenüber äußeren Einflüssen ist.