\section{Durchführung}
\label{sec:Durchführung}

\subsection{Justage}
Das Sagnac-Interferometer wird, wie in \autoref{fig:aufbau} dargestellt aufgebaut.
Im vom PBSC ausgehenden Strahl befindet sich ein Polarisator und anschließend ein Schirm, auf dem das Interferenzmuster abgebildet wird.
\\
Ziel bei der Einstellung der Spiegel ist, dass diese mittig und unter einem Winkel von $\qty{45}{\degree}$ getroffen werden.
Justageplatten mit einer kleinen Öffnung können dazu vor die Spiegel oder den PBSC platziert werden.
Die Position des Spiegels wird dann so gewählt, dass der Lichtstrahl die Öffnung durchläuft und den Spiegel mittig trifft.
\\
Die Spiegelpositionen können teilweise nur horizontal oder vertikal mit einer Feinjustierschraube variiert werden.
Mithilfe von Justageplättchen wird die horizontale Komponente der vertikal verstellbaren Spiegel angepasst.
Metallplättchen werden dazu unter die Bodenplatte der Spiegel platziert und ermöglichen den Winkel der Spiegelebene zu variieren.
\\
In den Rotationshalter wird die Doppelglasplatte eingesetzt.
Unter Beobachtung des Interferenzmusters werden die Spiegelpositionen angepasst.
Das Doppelglas kann in einem Winkel bis $\qty{10}{\degree}$ gedreht werden und dabei das Interferenzmuster beobachtet werden.
Es sollten im Interferenzmuster keine Streifen mehr zu sehen sein.
Wenn weiter nur ein heller bzw. dunkler Punkt auf dem Schirm zu sehen ist, verlaufen beide Teilstrahlen im Sagnac-Interferometer auf der gesamten Länge parallel zueinander.
Die Justage ist somit abgeschlossen.

\subsection{Kontrastbestimmung}
Bevor die Messung der Brechungsindizes beginnen kann, wird der Kontrast des Interferometers maximiert.
Dazu wird in Abhängigkeit der Polarisation des auf den PBSC treffenden Strahls der Kontrast bestimmt.
Der Winkel des Polarisationsfilters wird dazu im Bereich von $\qty{0}{\degree}$ bis $\qty{180}{\degree}$ in $\qty{15}{\degree}$-Schritten variiert.
Nahe des Maximums wird die Schrittweite auf $\qty{5}{\degree}$ reduziert, um eine höhere Auflösung im relevanten Bereich zu erzielen.
\\
Der Kontrast ergibt sich aus den Intensitätsmaxima und -minima.
Dazu wird der Schirm durch eine Photodiode ausgetauscht.
Die Ausgangsspannung ist proportional zu der gemessenen Intensität des interferierenden Strahls und wird mit einem Multimeter gemessen.
\\
Weitere Messungen werden unter dem Polarisationswinkel durchgeführt, für den der Kontrast maximal ist.

\subsection{Brechungsindex von Glas}
Die Messung der Brechungsindizes wird über die Differenzspannungsmethode durchgeführt.
Der Polarisationsfilter vor der Photodiode wird dazu durch einen weiteren um $\qty{45}{\degree}$ gedrehten PBSC ersetzt.
Über zwei Photodioden werden die senkrecht und parallel polarisierten Teilstrahlen detektiert.
\\
Beide Photodioden sind an die Eingänge des Modern Inferometry Controller angeschlossen.
Der Ausgang ist dabei proportional zu der Differenz der Eingangssignale und wird auf einem Oszilloskop dargestellt.
Ebenfalls detektiert der Modern Interferometry Controller die Anzahl an Nulldurchgängen.
Dies entspricht der Anzahl Interferenzmaxima bzw. -minima und tritt auf, wenn beide Dioden die gleiche Lichtintensität messen.
\\
Der Winkel des Doppelglashalter wird gleichmäßig von $\qty{0}{\degree}$ bis $\qty{10}{\degree}$ variiert.
Anschließend wird die Anzahl der Nulldurchgänge vom Modern Inferometry Controller abgelesen.
Die Messung wird 10 mal wiederholt.

\subsection{Brechungsindex von Gas}
Der Doppelglashalter wird aus dem Experiment entfernt und die Gaszelle in einen der Teilstrahle gesetzt.
Die Gaszelle wird zunächst mit einer Pumpe evakuiert.
Über ein Luftventil kann die Menge der einströmenden Luft angepasst werden.
Der Druck wird mit einem Digitalmanometer überwacht.
\\
In $\qty{50}{\milli\bar}$-Schritten wird die Anzahl der Nulldurchgänge in Abhängigkeit des Luftdrucks aufgenommen bis wieder Normaldruck erreicht ist.
Es werden drei Messreihen aufgenommen.