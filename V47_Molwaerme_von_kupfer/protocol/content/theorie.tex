\section{Theory}
\label{sec:Theory}

\subsection{Heat capacity}
\label{ssec:theory1}

The amount of heat $\Delta Q$ needed to rise the temperature $T$ by $\SI{1}{\kelvin}$ is called heat capacity and is defined as
\begin{equation}
    C = \frac{\Delta Q}{\Delta T}.
    \label{eq:heat}
\end{equation}
It is highly dependend on the amount of substance the heat is beeing transferred to.
To get a value that is independent of the amount of substance we define
\begin{equation}
    c^{\text{m}} = \frac{\Delta Q}{\Delta T \cdot \si{\mol}}.
    \label{eq:heat_mole}
\end{equation}

Futhermore there are two different definitions of $C$.
One where the volume of the body is consant $(\mathrm{d} V = 0)$ and one where the pressure is constant $(\mathrm{d} p = 0)$.
They are defined as 
\begin{align}
    C_\text{V} &= \left. \frac{\Delta U}{\Delta T} \right\vert_{V}\\
    C_\text{p} &= \left. \frac{\Delta Q}{\Delta T} \right\vert_{p}.
\end{align}
When measuring the heat capacity during constant pressure, a part of the heat energy is used to increase the volume of the body.
Because of that $C_\text{p}$ is always greater than $C_\text{V}$.
The difference between both heat capacities is given as
\begin{equation}
    C_\text{p} - C_\text{V} = 9 \alpha ^2 \kappa V_\text{0} T.
    \label{eq:heat_dif}
\end{equation}
Where $\alpha$ is the coefficient of thermal expansion, $\kappa$ is the bulk modulus and $V_\text{0}$ the volume of one mol.

\subsection{Classical model}
\label{ssec:theory2}

In the classical model we assume that there is a system with $3 N$ vibration modes.
With $N$ being the number of all atoms in the body.
The energy is equally split up in cinetic and potenial energy for each mode.
After that we get the internal energy from which we derive the Dulong–Petit law
\begin{equation}
    C_\text{V} = 3 N k_\text{B}.
    \label{eq:dulong}
\end{equation}

This model provides accurate results in high temperature experiments but fails to predict the low temperature area of $C_\text{V}$.
With the addition of quantum mechanics it is possible to get a more precise depiction of heat capacity.


\subsection{Einstein model}
\label{ssec:theory3}

The Einstein model is based on the assumption that all the eigenmodes have the same frequency $\omega_\text{E}$.
With that the densitiy of states can be defined as $D (\omega) = 3 N \delta (\omega - \omega_\text{E})$.
In addition the so called Einstein temperature 
\begin{equation}
    \theta_\text{E} = \frac{\hbar \omega_\text{E}}{k_\text{B}}
    \label{eq:einstein}
\end{equation}
is used to determine in which ranges of temperature the model works well.
Because the assumption can only be made, if the optical modes are dominant.
So the heat capacity is given for different temperatures ranges as
\begin{equation}
    C_\text{V}^\text{E} =
      \begin{cases}
        3 N k_\text{B} \left( \frac{\Theta_\text{E}}{T}   \right)^2 \exp{- \frac{\Theta_\text{E}}{T}} \text{für} T $<<$ \Theta_\text{E}\\
        3 N k_\text{B} \text{für} T $>>$ \Theta_\text{E}.\\
        \end{cases}       
\end{equation}
For high temperatures the Einstein model is the Dulong–Petit law.
Despite a more fitting result for low temperatures the observed $T^3$ dependency is not reached.

\subsection{Debye model}
\label{ssec:theory4}

The core assumption in the Debye model is that all phonons can be described via three modes with linear dispersion $\omega_\text{i} = \nu_\text{i} q$
