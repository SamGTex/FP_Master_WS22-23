\section{Theory}
\label{sec:Theory}

\subsection{Heat capacity}
\label{ssec:theory1}

The amount of heat $\Delta Q$ needed to rise the temperature $T$ by $\SI{1}{\kelvin}$ is called heat capacity and is defined as
\begin{equation}
    C = \frac{\Delta Q}{\Delta T}.
    \label{eq:heat}
\end{equation}
It is highly dependend on the amount of substance the heat is beeing transferred to.
To get a value that is independent of the amount of substance we define
\begin{equation}
    c^{\text{m}} = \frac{\Delta Q}{\Delta T \cdot \si{\mol}}.
    \label{eq:heat_mole}
\end{equation}

Futhermore there are two different definitions of $C$.
One where the volume of the body is consant $(\mathrm{d} V = 0)$ and one where the pressure is constant $(\mathrm{d} p = 0)$.
They are defined as 
\begin{align}
    C_\text{V} &= \left. \frac{\Delta U}{\Delta T} \right\vert_{V}\\
    C_\text{p} &= \left. \frac{\Delta Q}{\Delta T} \right\vert_{p}.
\end{align}
When measuring the heat capacity during constant pressure, a part of the heat energy is used to increase the volume of the body.
Because of that $C_\text{p}$ is always greater than $C_\text{V}$.
The difference between both heat capacities is given as
\begin{equation}
    C_\text{p} - C_\text{V} = 9 \alpha ^2 \kappa V_\text{0} T.
    \label{eq:heat_dif}
\end{equation}
Where $\alpha$ is the coefficient of thermal expansion, $\kappa$ is the bulk modulus and $V_\text{0}$ the volume of one mol.

\subsection{Classical model}
\label{ssec:theory2}

\subsection{Einstein model}
\label{ssec:theory3}

\subsection{Debye model}
\label{ssec:theory4}