\section{Discussion}
\label{sec:Diskussion}

The mean of all calculated Debye's temperatures is determined to
\begin{equation}
    \theta_\text{D} = \qty{205.26(16)}{\kelvin}.
\end{equation}
With that the relative deviation is about $\qty{38}{\percent}$.
One reason for this difference is that the Debye's temperature is calculated via the first measurements.
In theory those are raising steadily but there are several outliers. 
The temperatures of the sample and the cylinder are supposed to be the same, but it turned out to be rather difficult to adjust the current.
The reason was the temperature's slow behaviour as the current changed.
In particular the first value could only be measured with big uncertainties, because of a short heating time.
\\
\\
To achieve better results, the measurement could be automated. 
At the end of every time period, 5 measured values had to be recorded. 
These could be read out instantaneously and the error of the measured time could be significantly reduced.