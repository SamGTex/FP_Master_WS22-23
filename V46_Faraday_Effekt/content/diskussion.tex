\section{Diskussion}
\label{sec:Diskussion}
Zur Bestimmung des externen Magnetfelds das auf die Galliumarsenid-Probe wirkt, wurde angenommen dass die Probe am Ort mit maximaler Feldstärke positioniert ist.
Diese Annahme konnte bei der Messung bestätigt werden.
Am Ort der Probe wurde das maximale Magentfeld gemessen.
\\
Die effektive Masse der ersten n-dotierten Probe mit Ladungsträgerkonzentration $N_1 = \qty{1.2e18}{\centi\metre^2}$ bzw. der zweiten Probe mit $N_2 = \qty{2.8e18}{\centi\metre^2}$ wird auf 
\begin{align*}
    m^*_{N_1} &= \qty{0.086(9)}{m_e}, \\
    m^*_{N_2} &= \qty{0.081(4)}{m_e}
\end{align*}
bestimmt.
Im Vergleich zum Literaturwert $m^*_\text{lit} = \qty{0.067}{m_e}$ \cite{eff_m} weicht die experimentell bestimmte effektive Masse der ersten Probe um $\Delta m^*_{N_1} = \qty{29}{\percent}$ und der zweiten Probe um $\Delta m^*_{N_2} = \qty{20}{\percent}$ ab.
Die Abweichungen können auf Messunsicherheiten zurückgeführt werden.
Bei der Messung des Rotationswinkels $\theta$ wird der Winkel des Polarisationsfilter so eingestellt, dass am Oszilloskop die minimale Amplitude erscheint.
Diese Methode weist eine große Unsicherheit im gemessenen Winkel auf, da das Minimum häufig nicht eindeutig zu bestimmen ist.
\\
Dies ist auch in den Messwerten, insbesondere für die n-dotierte Probe mit Ladungsträgerkonzentration $N_2$ zu erkennen.
Exemplarisch stellt der gemessenen Winkel $\theta$ bei der Wellenlänge $\qty{2.51}{\micro\metre}$ einen Ausreißer dar (siehe \autoref{fig:theta_dif_lam2}) und passt nicht zu dem erwarteten Verlauf.
Eine verbesserte Justierung des Strahlengangs und die Optimierung der Parameter am Selektivverstärker könnte das Signal verbessern und somit zu eindeutigen Minima führen.