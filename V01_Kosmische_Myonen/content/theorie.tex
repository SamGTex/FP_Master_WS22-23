\section{Theorie}
\label{sec:Theorie}

\subsection{Das Myon}
\label{ssec:myon}

Myonen sind Elementarteilchen des Standardmodells der Teilchenphysik.
Die Fermionen des Standardmodells sind in drei Generationen eingeteilt, wobei das Myon der 2. Generation angehört.
Es ähnelt dabei stark dem Elektron aus der 1. Generation. 
So besitzen beide Teilchen einen Spin $S = \frac{1}{2}$ und eine Ladung von $q = -e$.
Das Myon ist mit einer Masse von $m = \qty{105.658}{\MeV}$ über 200 mal schwerer als das Elektron.

Die Myonen die auf der Erdoberfläche ankommen stammen aus Wechselwirkungen der primären Kosmischen Strahlung mit der Atmosphäre.
Dort entstehen größtenteils geladene Pionen, die wiederum wie folgt zerfallen:
\begin{align*}
    \pi^- &\rightarrow \mu^- + \bar{\nu}_{\mu} & \pi^+ &\rightarrow \mu^+ + \nu_{\mu}
\end{align*}
Die entstandenen Myonen haben eine durchschnittliche Lebensdauer von $\tau = \qty{2.197}{\micro\second}$.
Myonen zerfallen über:
\begin{align*}
    \mu^- &\rightarrow \symup{e}^- + \bar{\nu}_\text{e} + \nu_{\mu} & \mu^+ &\rightarrow \symup{e}^+ + \nu_{\text{e}} + \bar{\nu}_{\mu}
\end{align*}
\subsection{Lebensdauer}
\label{ssec:lebensdauer}

Alle instabilen Teilchen besitzen notwendigerweise eine mittlere Lebensdauer.
Nach dieser Zeit zerfallen sie in andere Teilchen.
Allgemein ist die Wahrscheinlichkeit, dass ein Teilchen in einer Zeit $\text{d}t$ zerfällt 
\begin{equation}
    \text{d}W = \lambda \text{d}t.
\end{equation}
Dabei ist $\lambda$ die Zerfallskonstante für diesen Zerfall.
Wird diese Überlegung auf eine Anzahl $N$ Teilchen angewendet, definieren wir die Änderung der Teilchenzahl $\text{d}N$.
Die Änderung der Teilchenzahl in der Zeit $\text{d}t$ ist dann definiert als
\begin{equation}
    \text{d}N = - N \lambda \text{d}t.
\end{equation}
Diese Differentialgleichung kann mit dem Ansatz 
\begin{equation}
    N(t) = N_0 \exp{- \lambda t}
\end{equation}
gelöst werden.
Damit erhalten wir das Zerfallsgesetz, welches zu jedem Zeitpunkt $t$ angibt, wie viele Teilchen von den ursprünglich $N_0$ noch nicht zerfallen sind.
Aus der Zerfallskonstante kann dann die mittlere Lebensdauer $\tau = \frac{1}{\lambda}$ bestimmt werden.