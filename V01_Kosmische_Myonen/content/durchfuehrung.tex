\section{Durchführung}
\label{sec:Durchführung}

Bevor die komplette Schaltung aufgebaut wird, werden die einzelnen Bauteile überprüft und kalibriert.
Wenn die Photomultiplier mit Strom versorgt werden, sollten am Oszilloskop Signale erkennbar sein.
Diese Signalrate wird im Folgenden über die Schwellspannungen an den Diskriminatoren auf etwa $30$ Signale pro $\unit{\second}$ eingestellt.
Um die Impulse zu zählen wird zusätzlich ein Impulszähler angeschlossen.
Die Pulsdauer der Diskriminatoren kann am Oszilloskop überprüft werden und auf etwa $\Delta = \qty{10}{\nano\second}$ festgelegt.

Im nächsten Schritt wird die Schaltung bis zur Koinzidenz aufgebaut und der dortige Ausgang überprüft.
Dafür wird das Ausgangssignal der Koinzidenz an einen Impulszähler angeschlossen und systematisch die Verzögerung an der Verzögerungsleitung erhöht.
Wichtig ist hierbei, dass der Messbereich groß genug ist, damit die Halbwertszeit bestimmt werden kann.
Nach der Messung wird eine Verzögerung ausgewählt, bei der die Signalrate etwa $20$ Signale pro $\unit{\second}$ beträgt.

Der restliche Teil der Schaltung wird gemäß \autoref{fig:szintillator} aufgebaut.
Allerdings wird der Doppelimpulsgenerator an den Vielkanalanalysator angeschlossen, damit dieser kalibriert werden kann.
Am Doppelimpulsgenerator wird eine Zeit zwischen den beiden Impulsen ausgewählt, die die Zerfallszeit simuliert.
Damit kann überprüft werden welche Zerfallszeit in welches Bin des Vielkanalanalysators eingeordnet wird.
Es sollen mindestens zehn Messwerte im Bereich $\qty{0.3}{\micro\second}$ bis $\qty{9.9}{\micro\second}$ aufgenommen werden.
Nach dieser Messung werden die Photomultiplier wieder angeschlossen und die Schaltung sollte nun \autoref{fig:szintillator} entsprechen.
Die eigentliche Messung der Lebenszeit läuft nun für etwa $\qty{20}{\hour}$ bis $\qty{30}{\hour}$.