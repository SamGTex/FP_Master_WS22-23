\section{Auswertung}
\label{sec:Auswertung}

%Siehe \autoref{fig:plot}!
\subsection{Verzögerungszeit}
Im Folgenden wird für Messgrößen, die eine Anzahl beschreiben eine Poisson-verteilte Messunsicherheit $\sigma = \sqrt{N}$ angenommen.
In weiteren Rechnungen wird mithilfe der Python-Bibliothek \textit{uncertainties}\cite{uncertainties} die Fehlerfortpflanzung durchgeführt.
\\
Die in einem Zeitraum von $\qty{10}{\second}$ gemessenen Impulse $N$ bei eingestellter Verzögerungszeit $t$ sind in \autoref{tab:verzoegerung} aufgelistet.
Eine Verzögerung auf der linken bzw. rechten Leitung entspricht einer negativen bzw. positiven Verzögerungszeit.
\begin{table}
    \centering
    \caption{Die Anzahl detektierter Ereignisse $N$ in Abhängigkeit der eingestellten Verzögerungszeit $t$ der Verzögerungsleitungen.
    Eine Verzögerung auf der linken bzw. rechten Leitung entspricht einer negativen bzw. positiven Verzögerungszeit.
    Die Messzeit beträgt jeweils $\qty{10}{\second}$.
    }
    \label{tab:verzoegerung}
    \begin{tabular}{cc|cc}
        \toprule
        $t \,/\, \unit{\nano\second}$ & Anzahl Impulse $N$ in $\qty{10}{\second}$ & $t \,/\, \unit{\nano\second}$ & Anzahl Impulse $N$ in $\qty{10}{\second}$ \\
        \midrule
        -28 & 0 & 1 & 215 \\
        -26 & 0 & 2 & 227 \\
        -24 & 0 & 3 & 237 \\
        -22 & 0 & 4 & 260 \\
        -20 & 1 & 6 & 281 \\
        -18 & 9 & 8 & 300 \\
        -16 & 10 & 10 & 255 \\
        -14 & 23 & 12 & 234 \\
        -12 & 35 & 14 & 162 \\
        -10 & 57 & 16 & 96 \\
        -8 & 99 & 18 & 75 \\
        -6 & 168 & 20 & 45 \\
        -4 & 183 & 22 & 25 \\
        -3 & 188 & 24 & 20 \\
        -2 & 205 & 26 & 10 \\
        -1 & 221 & 28 & 6 \\
        0 & 212 & & \\
        \bottomrule
    \end{tabular}
\end{table}
\begin{figure}
    \centering
    \includegraphics[width=0.9\textwidth]{content/plots/verzoegerungszeit.pdf}
    \caption{Die gemessene Ereignisrate $N$ in Abhängigkeit der Verzögerungszeit $\Delta t$.
    Die Messwerte werden in zwei Bereiche unterteilt, wobei die ersten und letzten Messwerte nicht berücksichtigt werden.
    Der Anstieg und Abfall wird näherungsweise jeweils durch einen linearen Fit dargestellt.}
    \label{fig:verzoegerung}
\end{figure}
\\
Die Messwerte werden in einen Anstiegs- und Abstiegsbereich unterteilt.
Wird das Signal eines Photomultipliers zu stark verzögert, löst die Koinzidenz kein Signal aus.
Im mittleren Bereich wäre ein Plateau zu erwarten, dieses wird in diesen Daten nicht beobachtet.
\\
Statt die Gesamtanzahl gemessener Impulse in einem bestimmten Zeitraum wird die Impulsrate pro Sekunde betrachtet.
Anstiegs- und Abstiegsbereich folgen einem linearen Verlauf und daher wird ein linearer Fit der Form
\begin{equation*}
    N = a \cdot t + b
\end{equation*}
mithilfe der Methode der kleinsten Quadrate (\textit{scipy.optimize.curve\_fit}\cite{scipy}) durchgeführt.
Aus der Ausgleichsrechnung folgen die Parameter
\begin{align*}
    a_\text{Anstieg} &= \qty{1.37(6)e9}{\frac{1}{\second^2}},\qquad b_\text{Anstieg} = \qty{21.08(61)}{\frac{1}{\second}}, \\
    a_\text{Abfall} &= \qty{-1.80(20)e9}{\frac{1}{\second^2}},\qquad b_\text{Abfall} = \qty{41.27(383)}{\frac{1}{\second}} \,.
\end{align*}
Das Maximum ergibt sich als Schnittpunkt der Geraden und kann auf 
\begin{equation}
    N_\text{Max} = \qty{29.81(72)}{\frac{1}{\second}}
\end{equation}
bestimmt werden.
Aus den Schnittpunkten der Ausgleichsgeraden mit der halben maximalen Höhe $N_\text{Max}$
\begin{align*}
    t_\text{Anstieg} &= \qty{-4.49(40)}{\nano\second} \\
    t_\text{Abfall} &= \qty{14.62(267)}{\nano\second}
\end{align*}
kann die Halbwertsbreite als Differenz der Schnittpunkte zu
\begin{equation}
    \Delta t = t_\text{Abfall} - t_\text{Anstieg} = \qty{19.11(270)}{\nano\second}
\end{equation}
bestimmt werden.
Die Messwerte und Ausgleichsgeraden sind in \autoref{fig:verzoegerung} dargestellt.
Um im Folgenden eine möglichst hohe Impulsrate zu messen, wird die Verzögerungszeit auf $\Delta t = \qty{8}{\nano\second}$ eingestellt.%???
\FloatBarrier

\subsection{Kalibration des Vielkanalanalysators}
Der Vielkanalanalysator ordnet ein Ereignis einem Kanal zu.
Jeder Kanal deckt ein definiertes Zeitintervall ab.
Die mittlere Zeit jedes Kanals wird mithilfe eines Impulsgenerators bestimmt.
\\
Die ermittelten Kanäle für die eingestellten Zeitabstände zwischen zwei Impulsen sind in \autoref{tab:kalibration} gelistet.
\begin{table}
    \centering
    \caption{Die ermittelten Kanäle am Vielkanalanalysator in Abhängigkeit des eingestellten Impulsabstands am Impulsgenerator.}
    \label{tab:kalibration}
    \begin{tabular}{cc}
        \toprule
        $t \,/\, \unit{\micro\second}$ & Kanal \\
        \midrule
        0.5 & 4 \\
        1 & 16 \\
        2 & 38 \\
        3 & 61 \\
        4 & 84 \\
        5 & 107 \\
        6 & 129 \\
        7 & 152 \\
        8 & 175 \\
        9 & 197 \\
        9.9 & 218 \\
        \bottomrule
    \end{tabular}
\end{table}
Die Kanäle sind mit dem Impulsabstand linear korreliert.
Ein linearer Fit der Form
\begin{equation}
    t = a \cdot ch + b
    \label{eqn:fit_kanal}
\end{equation}
wird durchgeführt, wobei $ch$ die Kanäle ("`channel"') im Bereich $[0, 511]$ und $t$ den Impulsabstand beschreibt.
Die Ausgleichsgerade mit den Parametern
\begin{align*}
    a_\text{Kalibration} = \qty{0.0440(00001)}{\second}, \qquad b_\text{Kalibration} = \qty{0.3125(00079)}{\second}
\end{align*}
und die Messwerte sind \autoref{fig:kalibration} abgebildet.
\begin{figure}
    \centering
    \includegraphics[width=0.9\textwidth]{content/plots/calibration.pdf}
    \caption{Die ermittelten Kanäle des Vielkanalanalysators in Abhängigkeit des eingestellten Impulsabstands.
    Ein linearer Fit verdeutlicht die lineare Korrelation zwischen den Variablen.
    }
    \label{fig:kalibration}
\end{figure}
Jedem Kanal kann jetzt eine Zeit zugeordnet werden.
\FloatBarrier

\subsection{Statistische Abschätzung der Untergrundereignisse}
\label{sec:untergrund}
Die eigentliche Messung der Lebensdauer der Myonen umfasst eine Messzeit von
\begin{equation*}
    t_\text{mess} = \qty{158234}{\second} \,.
\end{equation*}
Die Gesamtanzahl der Start- $N_\text{Start}$ und Stoppsignale $N_\text{Stop}$ beträgt
\begin{align*}
    N_\text{Start} &= \qty{4509112}{}, \\
    N_\text{Stop} &= \qty{17526}{} \,.
\end{align*}
Durch ein eintreffendes Myon wird das Startsignal ausgelöst.
Zerfällt das Myon in der Suchzeit $T_\text{S} = \qty{10}{\micro\second}$ wird ein weiteres Signal, das Stoppsignal detektiert.
Im Folgenden wird die Untergrundrate mit statistischen Methoden abgeschätzt.
Die Ereignisrate beträgt im Durchschnitt etwa $n = \qty{28.5}{\frac{1}{\second}}$.
Die Wahrscheinlichkeit, dass während der Suchzeit $T_\text{S}$ ein weiteres Myon in den Detektor trifft folgt einer Poisson-Verteilung.  
Der Untergrund wird mittels \autoref{eqn:poisson_untergrund} auf
\begin{equation}
    N_\text{Untergrund} = \qty{1283}{}
    \label{eqn:untergrund}
\end{equation}
abgeschätzt.
Dabei verteilt sich der Untergrund auf alle Kanäle.
\FloatBarrier

\subsection{Lebensdauer kosmischer Myonen}
Die Lebensdauer der Myonen $\tau$ wird mithilfe einer exponentiellen Ausgleichskurve bestimmt.
Zuerst wird der Untergrund als Fit-Variable $U_1$ behandelt.
Es wird eine Ausgleichsrechnung mit der Funktion
\begin{equation}
    N(t) = N_0 \cdot e^{- \lambda t} + U_1
\end{equation}
durchgeführt.
\begin{figure}
    \centering
    \includegraphics[width=0.9\textwidth]{content/plots/lifetime.pdf}
    \caption{Die Anzahl detektierter Ereignisse $N$ die ein Start- und Stoppsignal ausgelöst haben in Abhängigkeit der gemessenen Zerfallszeit $t$.
    Die exponentielle Ausgleichskurve führt direkt zur Lebensdauer der kosmischen Myonen.
    }
    \label{fig:lebensdauer1}
\end{figure}
Dabei werden die Kanäle im Bereich $[4, 218]$ beachtet.
Die ersten vier Kanäle weisen eine sehr geringe Ereignisrate auf, obwohl im Anfangsbereich hohe Ereignisraten erwartet werden.
Ein möglicher Grund ist, dass der Vielkanalanalysator oder der TAC zu kurze Zerfallszeiten nicht erfassen kann.
Setzt man den Kanal $218$ in \autoref{eqn:fit_kanal} ein, so erhält man eine Zerfallszeit von $\sim \qty{10}{\micro\second}$.
Dies entspricht der Suchzeit $T_S$.
Alle weiteren Kanäle weisen daher sehr geringe bis keine Ereignisse auf und dürfen nicht in der Ausgleichsrechnung berücksichtigt werden.
\\
Die betrachteten Kanäle werden analog in die entsprechenden Zerfallszeiten umgerechnet und anschließend der Fit durchgeführt.
Dabei ergibt sich für die Lebensdauer $\tau = 1/\lambda$ und für den Untergrund $U_1$
\begin{align}
    \tau_1 &= \qty{2.11(5)}{\micro\second},\\
    U_1 &= \qty{1.05(41)}{} \,.
\end{align}
Die Messwerte und exponentielle Ausgleichskurve sind in \autoref{fig:lebensdauer1} dargestellt.
\\
\\
Zuletzt wird der in \autoref{sec:untergrund} statistisch abgeschätzte Untergrund zur Bestimmung der Lebensdauer der kosmischen Myonen verwendet.
In der exponentiellen Ausgleichsfunktion
\begin{equation}
    N(t) = N_0 \cdot e^{- \lambda t} + U_2
    \label{eqn:exp2}
\end{equation}
wird der Untergrund nicht weiter als Fit-Parameter behandelt.
$U_2$ ist hier der statistisch ermittelte Untergrund.
Es wird angenommen, dass sich der Untergrund (\autoref{eqn:untergrund}) gleichmäßig auf alle Kanäle verteilt.
Dabei werden weiter die Kanäle $[4,218]$ betrachtet.
Der Untergrund pro Kanal ergibt sich dann zu
\begin{equation}
    U_2 = \qty{5.97}{} \, .
\end{equation}
Aus einer Ausgleichsrechnung mit \autoref{eqn:exp2} folgt
\begin{equation}
    \tau_2 = \qty{1.69(4)}{\micro\second}
\end{equation}
für die Lebensdauer der Myonen unter Beachtung des statistich abgeschätzten Untergrunds.
Die Anzahl Ereignisse $N$ für die Zerfallszeiten der Kanäle $[4,218]$, sowie die Ausgleichskurve sind in \autoref{fig:lebensdauer2} dargestellt.
\begin{figure}
    \centering
    \includegraphics[width=0.9\textwidth]{content/plots/lifetime2.pdf}
    \caption{Die Anzahl detektierter Ereignisse $N$ die ein Start- und Stoppsignal ausgelöst haben in Abhängigkeit der gemessenen Zerfallszeit $t$.
    Die exponentielle Ausgleichskurve berücksichtigt den statistisch ermittelten Untergrund als Konstante.
    }
    \label{fig:lebensdauer2}
\end{figure}