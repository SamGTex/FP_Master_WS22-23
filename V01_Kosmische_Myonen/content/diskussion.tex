\section{Diskussion}
\label{sec:Diskussion}
Bei der Untersuchung der Verzögerungszeit wurde nicht wie zu erwarten ein Plateau gemessen.
Die bestimmte Halbwertsbreite $\Delta t = \qty{19.11(270)}{\nano\second}$ entspricht nicht der eingestellten doppelten Impulsdauer $\Delta t_\text{th} = 2 \cdot \qty{20}{\nano\second}$.
Die Einstellung der Impulsdauer mittels Oszilloskop wurde nicht korrekt durchgeführt.
Beim Ablesen der Impulsdauer wurde lediglich auf den Abstand der Peaks geachtet und somit die Impulsbreite vernachlässigt.
Es wurde also nicht beachtet, dass das Signal eine Ausdehnung besitzt.
Eine bessere Möglichkeit wäre es gewesen, den Impulsabstand auf halber Höhe zu bestimmen.
\\
\\
Die gemessene Lebensdauer der kosmischen Myonen wurde über zwei Methoden bestimmt.
Unter Beachtung des Untergrunds als Parameter in der Ausgleichsrechnung ergibt sich für die Lebensdauer $\tau_1 = \qty{2.11(5)}{\micro\second}$.
Verglichen mit dem Literaturwert $\tau_\text{lit} = \qty{2.197}{\micro\second}$\cite{pdg} weicht die errechnete Lebensdauer um $\qty{3.96}{\percent}$ ab.
\\
Wird der Untergrund mittels statistischer Abschätzung berücksichtigt folgt für die Lebensdauer $\tau_2 = \qty{1.69(4)}{\micro\second}$.
Der über diese Methode errechnete Wert weist mit $\qty{23}{\percent}$ eine deutlich größere Abweichung zum Literaturwert auf.
Auffällig ist, dass der mittels Poisson-Verteilung abgeschätzte Untergrund $U_2 = \qty{5.97}{}$ verglichen mit dem mittels Ausgleichsrechnung bestimmte Untergrund $U_1 = \qty{1.05(41)}{}$ überschätzt wird.
Ein möglicher Grund ist, dass die Ausreißer nicht in der Ausgleichsrechung berücksichtigt werden, aber einen Beitrag zum Untergrund liefern können.
Um die statistische Aussagekraft der Daten zu erhöhen kann ein größerer Szintillatortank verwendet werden.
Ebenso kann die gesamte Messzeit des Experiments erhöht werden.