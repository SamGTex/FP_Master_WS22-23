\section{Diskussion}
\label{sec:Diskussion}
Die gemessene Lebensdauer der kosmischen Myonen wurde über zwei Methoden bestimmt.
Unter Beachtung des Untergrunds als Parameter in der Ausgleichsrechnung ergibt sich für die Lebensdauer $\tau_1 = \qty{2.11(5)}{\micro\second}$.
Verglichen mit dem Literaturwert $\tau_\text{lit} = \qty{2.197}{\micro\second}$\cite{pdg} weicht die errechnete Lebensdauer um $\qty{3.96}{\percent}$ ab.
\\
Wird der Untergrund mittels statistischer Abschätzung berücksichtigt folgt für die Lebensdauer $\tau_2 = \qty{1.93(4)}{\micro\second}$.
Der über diese Methode errechnete Wert weist mit $\qty{12.15}{\percent}$ eine größere Abweichung zum Literaturwert auf.
Auffällig ist, dass der mittels Poisson-Verteilung abgeschätzte Untergrund $U_2 = \qty{5.97(1)}{}$ verglichen mit dem mittels Ausgleichsrechnung bestimmte Untergrund $U_1 = \qty{1.05(41)}{}$ überschätzt wird.
Zur Vereinfachung wurde angenommen, dass sich der Untergrund gleichmäßig auf die angesprochenen Kanäle verteilt.
Diese Annahme ist nur näherungsweise gültig.
Die ersten Kanäle (geringe Zerfallszeiten) mit großen Zählraten $N$ weisen einen höheren Untergrund als die Kanäle mit einer geringen Ereignisrate auf.
Nach der Signalkorrektur (Abzug des Untergrunds) sind viele der Zählraten (ab $t \gtrsim \qty{7}{\micro\second}$) negativ und dürfen nicht bei der Ausgleichsrechnung berücksichtigt werden.
Somit werden die Zählraten nicht optimal korrigiert, das sich in der errechneten Lebensdauer zeigt.