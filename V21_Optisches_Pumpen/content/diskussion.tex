\section{Diskussion}
\label{sec:Diskussion}

Der Kernspin von Isotop $\ce{^{85}Rb}$ beträgt $2.50 \pm 0.04$ und entspricht somit dem Literaturwert.
Im Vergleich dazu weicht der Kernspin vom Isotop $\ce{^{87}Rb}$ mit $1.55 \pm 0.03$ um $\qty{3.35}{\percent}$ vom Theoriewert ab.

Das Isotopenverhältnis innerhalb des Gasgemischs beträgt $1:2$.
Der experimentell bestimmte Wert liegt bei $0.53$ und weicht somit um $\qty{5.66}{\percent}$ ab.

Das Erdmagnetfeld in Deutschland ist nach der Physikalisch-Technischen Bundesanstalt $\qty{49613.4}{\nano\tesla}$. \cite{erdfeld}
Der berechnete Wert beträgt $\qty{23600(1100)}{\nano\tesla}$ und besitzt somit eine Abweichung von $\qty{110.23}{\percent}$.
Obwohl die Berechnung des Erdmagnetfelds nicht das eigentliche Ziel des Versuch war, stellt dieser Wert eine gute Näherung dar.

Der Quotient der Landé-Faktoren beträgt in der Theorie $1.5$.
Aus den hyperbolischen Fits folgt $1.67 \pm 0.09$, was um $\qty{10.18}{\percent}$ von der Theorie abweicht.

Das Experiment ist sehr lichtempfindlich. Äußere Lichteinflüsse wurden nur durch eine Decke verhindert, die teilweise das Sonnenlicht nicht gut abgeschirmt hat.
So konnte mehrmals beobachtet werden, wie eine Änderung des Tageslichts eine Änderung im Signal verursacht hat.
Insgesamt liegen die Messwerte im erwarteten Bereich.