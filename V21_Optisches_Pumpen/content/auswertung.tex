\section{Auswertung}
\label{sec:Auswertung}

%Siehe \autoref{fig:plot}!

Für beide Resonanzstellen werden jeweils die Stromstärken für die Sweep-Spule $I_\text{s}$ und für die Horizontalspule $I_\text{h}$ in Abhängigkeit von der Frequenz $f$.
Mit den geometrischen Abmessungen der Spulen aus REF ZU DURCHFÜHRUNG werden zusätzlich die entsprechenden Magentfelder $B$ bestimmt.
Die Messwerte sind in \autoref{tab:f} aufgeschrieben.

\begin{table}
    \centering
    \caption{Horizontale Magnetfelder an den Resonanzstellen von beiden Isotopen mit eingestellten Stromstärken der Spulen.}
    \label{tab:f}
    \begin{tabular}{r r r r r r r}
        \toprule
        $f \,/\, \unit{\kilo\hertz}$ & $I_\text{1,s} \,/\, \unit{\ampere}$ & $I_\text{1,h} \,/\, \unit{\ampere}$ & $B_\text{1} \,/\, \unit{\milli\tesla}$ & $I_\text{2,s} \,/\, \unit{\ampere}$ & $I_\text{2,h} \,/\, \unit{\ampere}$ & $B_\text{2} \,/\, \unit{\micro\tesla}$\\
        \midrule
        $100 $ & $625.0\pm1.0$ & $734.0\pm1.0$ & $0.0\pm3.0$ & $0.0\pm3.0$  & $37.7\pm2.6$ & $44.3\pm2.6$ \\
        $200 $ & $440.0\pm1.0$ & $677.0\pm1.0$ & $30.0\pm3.0$ & $30.0\pm3.0$  & $52.9\pm2.6$ & $67.2\pm2.6$ \\
        $300 $ & $422.0\pm1.0$ & $775.0\pm1.0$ & $48.0\pm3.0$ & $48.0\pm3.0$  & $67.6\pm2.6$ & $88.9\pm2.6$ \\
        $400 $ & $296.0\pm1.0$ & $770.0\pm1.0$ & $72.0\pm3.0$  & $72.0\pm3.0 $ & $81.0\pm2.6$ & $109.6\pm2.6$ \\
        $500 $ & $306.0\pm1.0$ & $898.0\pm1.0$ & $90.0\pm3.0$  & $90.0\pm3.0 $ & $97.4\pm2.6$ & $133.1\pm2.6$ \\
        $600 $ & $322.0\pm1.0$ & $710.0\pm1.0$ & $102.0\pm3.0$ & $126.0\pm3.0$ & $108.9\pm2.6$ & $153.3\pm2.6$ \\
        $700 $ & $334.0\pm1.0$ & $698.0\pm1.0$ & $120.0\pm3.0$ & $150.0\pm3.0$ & $125.4\pm2.6$ & $173.7\pm2.6$ \\
        $800 $ & $380.0\pm1.0$ & $432.0\pm1.0$ & $132.0\pm3.0$ & $192.0\pm3.0$ & $138.7\pm2.6$ & $194.4\pm2.6$ \\
        $900 $ & $607.0\pm1.0$ & $554.0\pm1.0$ & $132.0\pm3.0$ & $210.0\pm3.0$ & $152.4\pm2.6$ & $217.6\pm2.6$ \\
        $1000$ & $783.0\pm1.0$ & $710.0\pm1.0$ & $144.0\pm3.0$ & $222.0\pm3.0$ & $173.5\pm2.6$ & $237.5\pm2.6$ \\
        \bottomrule
    \end{tabular}
\end{table}

\begin{table}
    \centering
    \caption{Periodendauer der Rabi-Oszillationen in Abhängigkeit der RF-Amplitude für beide Peaks.}
    \label{tab:A}
    \begin{tabular}{r r r}
        \toprule
        $A \,/\, \unit{\volt}$ & $T_\text{1} \,/\, \unit{\milli\second}$ & $T_\text{2} \,/\, \unit{\milli\second}$\\
        \midrule
        $1.0 $ & $4.70 $ & $6.00 $\\
        $1.5 $ & $3.16 $ & $4.40 $\\
        $2.0 $ & $2.40 $ & $3.40 $\\
        $2.5 $ & $1.92 $ & $2.80 $\\
        $3.0 $ & $1.68 $ & $2.40 $\\
        $3.5 $ & $1.44 $ & $2.00 $\\
        $4.0 $ & $1.26 $ & $1.90 $\\
        $4.5 $ & $1.12 $ & $1.50 $\\
        $5.0 $ & $0.99 $ & $1.48 $\\
        $6.0 $ & $0.84 $ & $1.20 $\\
        $8.0 $ & $0.60 $ & $0.88 $\\
        $10.0$ & $ 0.47$ & $0.72 $\\
        \bottomrule
    \end{tabular}
\end{table}